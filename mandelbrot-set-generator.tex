\documentclass{article}
\usepackage{graphicx}
\usepackage{listings}

\title{Algorithm for Generating the Mandelbrot Set}
\author{Marion Kipsang}
\date{\today}

\begin{document}

\maketitle

\section{Introduction}

The Mandelbrot set is a famous fractal that exhibits intricate and self-replicating patterns. It is generated by applying a mathematical iteration to points in the complex plane and determining whether the sequence remains bounded or tends to infinity. Points that remain bounded are considered part of the Mandelbrot set.

In this document, we outline the algorithm for generating the Mandelbrot set.

\section{Algorithm}

The Mandelbrot set generation algorithm can be outlined as follows:

\begin{enumerate}
\item Define the region in the complex plane that you want to visualize as the Mandelbrot set. This is typically done by specifying the boundaries for the real and imaginary axes.

\item Discretize the complex plane into a grid of points (pixels) within the specified region. Each pixel represents a complex number.

\item For each pixel, iterate the Mandelbrot iteration formula: $z = z^2 + c$, where $z$ is a complex number starting from $0$, and $c$ is the complex number corresponding to the pixel. The iteration is repeated until the sequence either becomes unbounded or reaches a maximum number of iterations.

\item Assign a color to each pixel based on the number of iterations required. Points that remain bounded within the maximum iterations are considered part of the Mandelbrot set and are colored black. Points that exceed the maximum iterations are colored with a gradient based on the number of iterations.

\item Repeat steps 3 and 4 for all pixels in the grid.

\item Save the visualization as an image file to display the Mandelbrot set.
\end{enumerate}

\section{Pseudocode}

Here is the pseudocode for the Mandelbrot set generation algorithm:

\begin{lstlisting}[language=Python]
# Define the region in the complex plane
center = complex(-0.5, 0.0)
zoom = 1.5

# Discretize the complex plane into a grid of pixels
for y in range(height):
    for x in range(width):
        # Calculate the complex number c corresponding to the pixel
        c = center + (x - width/2) / (zoom * width) + (y - height/2) / (zoom * height) * 1j
        
        # Initialize z to 0
        z = 0
        
        # Iterate the Mandelbrot iteration formula
        for iteration in range(max_iterations):
            z = z**2 + c
            if abs(z) > 2:
                break
        
        # Assign color to the pixel based on the number of iterations
        if iteration == max_iterations - 1:
            color = black
        else:
            color = gradient(iteration)
        
        # Set the color of the pixel in the visualization
        set_color(x, y, color)
\end{lstlisting}

\section{Conclusion}

The algorithm for generating the Mandelbrot set is based on iteratively applying the Mandelbrot iteration formula to points in the complex plane. By determining whether the sequence remains bounded or tends to infinity, we can visualize the Mandelbrot set with intricate and fascinating patterns.

\end{document}
